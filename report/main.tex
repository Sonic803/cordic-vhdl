\documentclass[a4paper,openright]{report}
% Tipo di documento. L'uso di twoside implica che i capitoli inizino sempre con la prima pagina a sinistra, eventualmente lasciando una pagina vuota nel capitolo precedente. Se questa cosa è fastidiosa, è possibile rimuoverlo. 

% Dimensione dei margini
\usepackage[a4paper,top=3cm,bottom=3cm,left=3cm,right=3cm]{geometry} 
% Dimensione del font
\usepackage[fontsize=12pt]{scrextend}
% Lingua del testo
\usepackage[italian,english]{babel}
% Lingua per la bibliografia
% \usepackage[fixlanguage]{babelbib}
% Codifica del testo
\usepackage[utf8]{inputenc} 
% Font mono (quello di default non supporta il grassetto)
\usepackage{courier}
% Encoding del testo
\usepackage[T1]{fontenc}
% Permette di generare testo fittizio. Mi è stato utile 
% per capire quale sarebbe stata l'impostazione del 
% testo nella pagina prima che scrivessi un determinato paragrafo
\usepackage{lipsum}

\usepackage[
backend=biber,
bibstyle=other/custom-numeric,
citestyle=numeric,
sorting=none,
]{biblatex}
% Sorgente della bibliografia
\addbibresource{chapters/Bibliografia.bib}

% Citazioni

% Per ruotare le immagini
\usepackage{rotating}
% Per cambiare i capitoli
\usepackage{titlesec}
% \titleformat{\chapter}[display]
  % {\normalfont\bfseries}{}{-100pt}{\huge}
\titleformat{\chapter}[hang]
  {\normalfont\bfseries\huge}{\thechapter}{1em}{\huge}
% Per mostrare nell'indice anche le subsubsection
\setcounter{tocdepth}{3}
% Per modificare l'header delle pagine 
\usepackage{fancyhdr}               

% Librerie matematiche
\usepackage{amssymb}
\usepackage{amsmath}
\usepackage{amsthm}         

% Uso delle immagini
\usepackage{graphicx}
% Uso dei colori
\usepackage[dvipsnames,svgnames,x11names]{xcolor}
% Uso dei listing per il codice
\usepackage{listings}          
% Per inserire gli hyperlinks tra i vari elementi del testo 
% \usepackage[hang, flushmargin]{footmisc}
% Diversi tipi di sottolineature
\usepackage[normalem]{ulem}

\usepackage{lstautogobble}  % Fix relative indenting
\usepackage{color}          % Code coloring
\usepackage{zi4}            % Nice font

%Aggiunti io?
\usepackage{lmodern}
\usepackage{ragged2e}
\usepackage{caption}
\usepackage[newfloat,outputdir=out]{minted}
\usepackage{tcolorbox}
\tcbuselibrary{minted,skins,breakable,xparse}
\usepackage{float}
\usepackage[skip=5pt]{parskip}
\usepackage{setspace}
\usepackage{hyphsubst}
\usepackage{microtype}
\usepackage{hyperref}
\usepackage{footnotebackref}

% dopo minted
\usepackage{csquotes}

% subfigures
\usepackage{subcaption}

%ps to pdf
% \usepackage{auto-pst-pdf}

% commento
\newcommand{\cna}[1]{\textcolor{red}{[\bfseries Avena: #1]}}
\newcommand{\cadm}[1]{\textcolor{red}{[\bfseries Trenitalia: #1]}}

% Modifica lo stile dell'header
\pagestyle{fancy}
\fancyhf{}
% \fancyhead[CE,CO]{\rightmark}
% \fancyhead[LE,RO]{\textbf{\thepage}}
\fancyhead[L]{\rightmark}
\fancyhead[R]{\textbf{\thepage}}
\fancyfoot{}
\setlength{\headheight}{17pt}

% Rimuove il numero di pagina all'inizio dei capitoli
\fancypagestyle{plain}{
  \fancyfoot{}
  \fancyhead{}
  \renewcommand{\headrulewidth}{0pt}
}

% Togliendo il commento al comando che segue, si inseriscono nella bibliografia anche le fonti presenti in Bibliography.bib ma non citati direttamente con il comando \cite

% Modifica dello stile dei riferimenti, con il testo in cyano
\definecolor{DarkGreen}{RGB}{20,80,40}
\definecolor{Unipi}{RGB}{00,85,143} 
\definecolor{term_bg_color}{RGB}{240,241,240} 
\hypersetup{
    colorlinks,
    linkcolor=Unipi,
    citecolor=Unipi,
    urlcolor=Unipi
}


\DeclareCiteCommand{\supercite}[\mkbibsuperscript]
  {\iffieldundef{prenote}
     {}
     {\BibliographyWarning{Ignoring prenote argument}}%
   \iffieldundef{postnote}
     {}
     {\BibliographyWarning{Ignoring postnote argument}}}
  {\usebibmacro{citeindex}%
   \bibopenbracket\usebibmacro{cite}\bibclosebracket}
  {\supercitedelim}
  {}


% Aggiunti definizioni, teoremi, linea e listing
\newtheorem{definition}{Definizione}[section]
\newtheorem{theorem}{Teorema}[section]
\providecommand*\definitionautorefname{Definizione}
\providecommand*\theoremautorefname{Teorema}
\providecommand*{\listingautorefname}{Listing}
\providecommand*\lstnumberautorefname{Linea}

\raggedbottom

\usepackage{xpatch}

\tcbuselibrary{listingsutf8} % Allows minted in tcolorbox

% -----------------------------------------------------------------

\begin{document}

\setlength{\fboxsep}{0pt}
\selectlanguage{english}

% \hyphenpenalty=10000
% \exhyphenpenalty=10000

% \loadspellchecklist[it][wordlist.txt]
% \setupspellchecking[state=start]

\setminted{
    linenos=true,
    numbersep=5pt,
    bgcolor=term_bg_color,
    frame=single,
    framesep=5pt,
    tabsize=2,
    breaklines=true,
    baselinestretch=0.9
}

\setlength{\abovecaptionskip}{-2pt} 
\setlength{\belowcaptionskip}{8pt} 

\newenvironment{code}
    {\captionsetup{type=listing}}
    {\par\noindent\ignorespacesafterend}

\SetupFloatingEnvironment{listing}{name=Listing, listname=List of Listings}

\usemintedstyle{sas}
% \usemintedstyle{unipi}
% \usemintedstyle{xcode}
% \usemintedstyle[output]{rrt}

% Interlinea
\setstretch{1.1}

\begin{titlepage}
\begin{figure}[!htb]

\begin{center}
{
    \includegraphics[keepaspectratio=true,scale=0.5]{images/Frontespizio/cherubinFrontespizio.eps}
}
\end{center}

\end{figure}

\begin{center}
    \LARGE{UNIVERSITÀ DI PISA}
\end{center}

\vspace{15mm}
\begin{center}
    {\LARGE{\bf CORDIC: Cartesian to  \\ \vspace{3mm} Polar Coordinate Transformation }}
\end{center}
\vspace{30mm}

\vfill
\hrulefill
\\\centering{\large{ANNO ACCADEMICO 2024/2025}}

\end{titlepage}
\stepcounter{page}

\tableofcontents


\chapter{Introduction}
\begin{center}
    \begin{circuitikz}
        % blocks
        \node[draw, rectangle, minimum width=6cm, minimum height=4cm, align=center] 
        (cordic) at (0,0) {CORDIC};
        
        \node[draw, rectangle, minimum width=2.5cm, minimum height=1cm, align=center]
        (lut) at (1.5,-1.2) {LUT \\ \(\arctan\)};
    
        % in
        \node[left] at ($(cordic.west) + (-1,1.5)$) {x};
        \draw[->] ($(cordic.west) + (-1,1.5)$) -- ++(1,0);
    
        \node[left] at ($(cordic.west) + (-1,0.75)$) {y};
        \draw[->] ($(cordic.west) + (-1,0.75)$) -- ++(1,0);
    
        \node[left] at ($(cordic.west) + (-1,0)$) {clk};
        \draw[->] ($(cordic.west) + (-1,0)$) -- ++(1,0);
    
        \node[left] at ($(cordic.west) + (-1,-0.75)$) {rst};
        \draw[->] ($(cordic.west) + (-1,-0.75)$) -- ++(1,0);
    
        \node[left] at ($(cordic.west) + (-1,-1.5)$) {start};
        \draw[->] ($(cordic.west) + (-1,-1.5)$) -- ++(1,0);
    
        % out
        \node[right] at ($(cordic.east) + (1,1)$) {\(\rho\)};
        \draw[->] ($(cordic.east) + (0,1)$) -- ++(1,0);
    
        \node[right] at ($(cordic.east) + (1,0)$) {\(\theta\)};
        \draw[->] ($(cordic.east) + (0,0)$) -- ++(1,0);
    
        \node[right] at ($(cordic.east) + (1,-1)$) {valid};
        \draw[->] ($(cordic.east) + (0,-1)$) -- ++(1,0);
    \end{circuitikz}
\end{center}

\vspace{10pt}
It is required to design a digital circuit for implementing the transformation from cartesian coordinates into polar coordinates using the CORDIC algorithm in Vectoring mode. It is implemented
with these recursive equations:
\begin{align*}
    x_{i+1} & = x_i - y_i \cdot d_i \cdot 2^{-i} \\
    y_{i+1} & = y_i + x_i \cdot d_i \cdot 2^{-i} \\
    z_{i+1} & = z_i - d_i \cdot \arctan(2^{-i})
\end{align*}
where \(d_i = -1\) if \(y_i > 0\), \(+1\) otherwise. After \(n\) iterations, the equations converge to:
\[
    x_n = A_n \cdot \sqrt{x_0^2 + y_0^2}
\]
\[
    y_n = 0
\]
\[
    z_n = z_0 + \arctan\left(\frac{y_0}{x_0}\right)
\]
\[
    A_n = \prod_{i=0}^n \sqrt{1 + 2^{-2i}}
\]
\section{Circuit Applications}
The CORDIC (COordinate Rotation DIgital Computer) algorithm is an iterative method for performing vector rotations and 
solving mathematical functions such as trigonometric, hyperbolic, exponential, logarithmic, and square root operations. 
It was introduced by Jack E. Volder in 1959 to simplify the computation of these functions in hardware with limited resources.
\\\\
CORDIC is widely used because it eliminates the need for multiplication and division, relying instead on shift and addition operations. 
This makes it highly efficient for hardware implementations, particularly in devices with limited computational power, 
such as embedded systems, calculators, and digital signal processors (DSPs).
\\\\
For example the Intel 8087, a floating-point coprocessor introduced in the early 1980s, utilized the CORDIC algorithm to perform 
efficient trigonometric and hyperbolic computations, such as sine, cosine, and arctangent, without relying on hardware multipliers.
Similarly, during the Apollo Lunar Module missions, a precursor concept to CORDIC was employed in the Apollo Guidance Computer (AGC) 
to perform real-time navigation calculations. The AGC used iterative methods to determine angles and distances for lunar landings, 
efficiently converting Cartesian spacecraft coordinates to polar forms to ensure precise trajectory adjustments during descent.

The CORDIC algorithm can operate in two different modes: \textbf{rotation mode} and \textbf{vectoring mode}
\begin{itemize}
    \item \textbf{Rotation mode:} rotates a vector by a specified angle, used for calculating trigonometric functions or vector transformations
    \item \textbf{Vectoring mode:} the input vector will be rotated towards the x-axis until the y-component is reduced to zero. After a fixed number of iterations, the modulus of the vector can be retrieved on the x-axis, while its phase corresponds to the angle of rotation.
\end{itemize}
In this context, this project will focus on developing the vectoring mode of the CORDIC algorithm to convert Cartesian coordinates into polar form.
\chapter{Architecture}


By default, the CORDIC algorithm converges only for input angles within the range \( (-99.7^\circ, 99.7^\circ) \).
To address this limitation and enable the algorithm to handle arbitrary input angles, we introduced an \textbf{initial correction step}. This step adjusts the input angle to bring it into the principal range of the algorithm. The adjustment ensures that the CORDIC algorithm works seamlessly for all input angles, not just those within the default convergence range.

\[
    x_0 = -y_{\text{input}} \cdot d_{\text{input}}
\]
\[
    y_0 =  x_{\text{input}} \cdot d_{\text{input}}
\]
\[
    z_0 = -\frac{\pi}{2} \cdot d_{\text{input}}
\]

Additionally, since the iterative process of the CORDIC algorithm introduces a scaling factor \( A_n \), we normalize the final result by dividing \( \rho \) (the magnitude) by \( A_n \).

\section{Data Representation}

To implement the CORDIC algorithm, we used fixed-point arithmetic for the input and intermediate values. Specifically:

\begin{itemize}
    \item For \( x \), \( y \), \( \rho \) we used 16-bit fixed-point representation with 8 bits allocated for the fractional part.
    \item For \( z \) and \( \theta \) we used a 16-bit fixed-point representation with 13 bits allocated for the fractional part. This decision was made because they always lie within the range \( [-\pi, \pi] \), and the additional fractional bits ensure high angular precision.
    \item For the intermediate calculations a 24-bit representation was used to minimize truncation errors during the iterative process.
          % TODO Scrivere meglio
    \item All the inputs and the outputs are limited to 16 bits due to the I/O pin constraints of the Zybo board, which restricts the width of the data buses and necessitate uniform bit-width for inputs and outputs.
\end{itemize}

The number of iterations for the CORDIC algorithm was set to 16, as this provides a high level of precision while balancing computational efficiency. Beyond 16 iterations, the precision gained decreases significantly.
\chapter{VHDL code}


\section{CORDIC}
The following code shows the implementation of the CORDIC in vectoring mode for Cartesian to polar conversion. This implementation supports the fixed step for accepting inputs in all 4 quadrants, and also normalizes $\rho$. As said in Section \ref{sec:algorithm_implementation}, we followed a Finite State Machine (FSM) approach where the control part uses a status register represented as \texttt{current\_state}.

\begin{code}
    \vhdlCode{../vhdl/src/CORDIC.vhd}
    \captionof{listing}{\texttt{CORDIC.vhd}}
    \label{code:vhdl}
\end{code}


\section{Atan LUT}
The values of the LUT table were calculated using the formula:
\[
    \text{LUT}_i = \left\lfloor \text{atan}(2^{-i} )* 2^{21} \right\rfloor
\]
\begin{code}
    \vhdlCode{../vhdl/src/ATAN_LUT.vhd}
    \captionof{listing}{\texttt{ATAN\_LUT.vhd}}
    \label{code:lut}
\end{code}

\chapter{Verification and testing}

\section{Testbench}

\chapter{Vivado results}
After completing the verification phase, the circuit design was synthesized and implemented using the Vivado Tool, specifically targeting the Zybo Zynq-7010 development board.

\section{Vivado Design flow}
For the implementation of the CORDIC algorithm in vectoring mode for Cartesian-to-Polar conversion, the Vivado design flow was employed using Xilinx/AMD Vivado software. This process involved RTL Elaboration, Synthesis, and Implementation on the target FPGA, along with the application of design constraints and the extraction of power and timing reports. To ensure accurate and reliable timing analysis, all combinational logic paths were structured to follow a Register-Logic-Register configuration.

\section{RTL}
Vivado produced a logic network made of:

% todo cambiare numeri
\begin{enumerate}
    \item 144 cells (e.g. multiplexers, DFFs, adders)
    \item 68 I/O ports ($16_{x} + 16_{y} + 16_{\rho} + 16_{\theta} + 1_{clk} + 1_{rst} + 1_{start} + 1_{valid}$)
    \item 802 nets (for connecting all the components)
\end{enumerate}

The RTL Analysis generated the Elaborated Design shown in Figure \ref{fig:schematic}, consistent with the expected structure of the system.

\begin{figure}[H]
    \centering
    \includegraphics[width=\textwidth, trim=0 200 0 200, clip]{./images/Vivado/rtl.pdf}
    \caption{Elaborated RTL design.}
    \label{fig:schematic}
\end{figure}

\section{Synthesis timing report}
\begin{figure}[H]
    \centering
    \includegraphics[width=\textwidth, trim=0 160 0 160, clip]{./images/Vivado/setup_synthesis.pdf}
    \caption{Figure showing the elaborated RTL design with the main critical paths for set-up-time-violation found in synthesis step highlighted in blue.}
    \label{fig:setup_synthesis}
\end{figure}

\begin{table}[H]
    \centering
    \small
    \captionsetup{skip=10pt} 
    \begin{tabular}{lrrrrrrr}
        \hline
        Name &  Slack &  Levels &  Routes & From      & To                 & Total Delay    \\
        \hline
        Path 1 &  12.78 &      10 &       6 & y\_t\_reg[14]/C  & ARG/A[23]       &         6.77   \\
        Path 2 &  12.78 &      10 &       6 & y\_t\_reg[14]/C  & ARG\_\_0/A[23]  &         6.77   \\
        Path 3 &  12.90 &       9 &       6 & y\_t\_reg[14]/C  & ARG/A[19]       &         6.66   \\
        Path 4 &  12.90 &       9 &       6 & y\_t\_reg[14]/C  & ARG\_\_0/A[19]  &         6.66   \\
        Path 5 &  13.02 &       8 &       6 & y\_t\_reg[14]/C  & ARG/A[15]       &         6.54   \\
        Path 6 &  13.02 &       8 &       6 & y\_t\_reg[14]/C  & ARG\_\_0/A[15]  &         6.54   \\
        Path 7 &  13.03 &      10 &       6 & y\_t\_reg[14]/C  & ARG/A[22]       &         6.53   \\
        Path 8 &  13.03 &      10 &       6 & y\_t\_reg[14]/C  & ARG\_\_0/A[22]  &         6.53   \\
        Path 9 &  13.08 &      10 &       6 & y\_t\_reg[14]/C  & ARG/A[21]       &         6.47   \\
        Path 10 &  13.08 &      10 &       6 & y\_t\_reg[14]/C  & ARG\_\_0/A[21]  &         6.47   \\
        \hline
    \end{tabular}
    \caption{Table showing data of the main critical paths found in synthesis step}
    \label{tab:setup_synthesis}
\end{table}
    
\begin{figure}[H]
    \centering
    \includegraphics[width=\textwidth, trim=0 160 0 160, clip]{./images/Vivado/hold_synthesis.pdf}
    \caption{Figure showing the elaborated RTL design with the main critical paths for hold-time-violation found in synthesis step highlighted in blue.}
    \label{fig:hold_synthesis}
\end{figure}  

\begin{table}[H]
    \centering
    \small
    \captionsetup{skip=10pt} 
    \begin{tabular}{lrrrrrr}
        \hline
        Name    & Slack & Levels & Routes  & From           & To               & Total Delay \\
        \hline
        Path 11 &  0.28 &      0 &       1 & z\_t\_reg[23]/C & z\_out\_reg[15]/D & 0.30       \\
        Path 12 &  0.28 &      0 &       1 & z\_t\_reg[8]/C  & z\_out\_reg[0]/D  & 0.31       \\
        Path 13 &  0.28 &      0 &       1 & z\_t\_reg[18]/C & z\_out\_reg[10]/D & 0.31       \\
        Path 14 &  0.28 &      0 &       1 & z\_t\_reg[19]/C & z\_out\_reg[11]/D & 0.31       \\
        Path 15 &  0.28 &      0 &       1 & z\_t\_reg[20]/C & z\_out\_reg[12]/D & 0.31       \\
        Path 16 &  0.28 &      0 &       1 & z\_t\_reg[21]/C & z\_out\_reg[13]/D & 0.31       \\
        Path 17 &  0.28 &      0 &       1 & z\_t\_reg[22]/C & z\_out\_reg[14]/D & 0.31       \\
        Path 18 &  0.28 &      0 &       1 & z\_t\_reg[9]/C  & z\_out\_reg[1]/D  & 0.31       \\
        Path 19 &  0.28 &      0 &       1 & z\_t\_reg[10]/C & z\_out\_reg[2]/D  & 0.31       \\
        Path 20 &  0.28 &      0 &       1 & z\_t\_reg[11]/C & z\_out\_reg[3]/D  & 0.31       \\
        \hline
    \end{tabular}
    \caption{Table showing the characteristics regarding critical paths for hold-time-violation found in synthesis step.}
    \label{tab:hold_synthesis}
\end{table}



\section{Implementation timing report}

\begin{figure}[H]
    \centering
    \includegraphics[width=\textwidth, trim=0 160 0 160, clip]{./images/Vivado/setup_implementation.pdf}
    \caption{Figure showing the elaborated RTL design with the main critical paths for set-up-time-violation found during implementation step highlighted in blue.}
    \label{fig:setup_implementation}
\end{figure}

\begin{table}[H]
    \centering
    \small
    \captionsetup{skip=10pt} 
    \begin{tabular}{lrrrrrr}
        \hline
        Name    & Slack & Levels & Routes  & From           & To               & Total Delay \\
        \hline
        Path 1  & 10.19 &      10 &       6 & counter\_reg[2]/C & ARG/A[20]        & 9.16       \\
        Path 2  & 10.38 &      10 &       6 & counter\_reg[2]/C & ARG\_\_0/A[20]   & 8.97       \\
        Path 3  & 10.43 &      10 &       6 & counter\_reg[2]/C & ARG\_\_0/A[21]   & 9.13       \\
        Path 4  & 10.43 &      10 &       6 & counter\_reg[2]/C & ARG/A[22]        & 8.92       \\
        Path 5  & 10.57 &       8 &       6 & counter\_reg[2]/C & ARG\_\_0/A[12]   & 8.78       \\
        Path 6  & 10.59 &       9 &       6 & counter\_reg[2]/C & ARG\_\_0/A[17]   & 8.97       \\
        Path 7  & 10.60 &       9 &       6 & counter\_reg[2]/C & ARG/A[16]        & 8.75       \\
        Path 8  & 10.61 &       8 &       6 & counter\_reg[2]/C & ARG\_\_0/A[13]   & 8.94       \\
        Path 9  & 10.61 &      10 &       6 & counter\_reg[2]/C & ARG/A[21]        & 8.94       \\
        Path 10 & 10.62 &      10 &       6 & counter\_reg[2]/C & ARG\_\_0/A[22]   & 8.73       \\
        \hline
    \end{tabular}
    \caption{Table showing the main critical paths for the setup-time-violation during the implementation step.}
    \label{tab:setup_implementation}
\end{table}


\begin{figure}[H]
    \centering
    \includegraphics[width=\textwidth, trim=0 160 0 160, clip]{./images/Vivado/hold_implementation.pdf}
    \caption{Figure showing the elaborated RTL design with the main critical paths for hold-time-violation found during implementation step highlighted in blue.}
    \label{fig:hold_implementation}
\end{figure}

\begin{table}[H]
    \centering
    \small
    \captionsetup{skip=10pt} 
    \begin{tabular}{lrrrrrr}
        \hline
        Name    & Slack & Levels   & From                              & To                    & Total Delay \\
        \hline
        Path 11 &  0.17 &       1 &  current\_state\_reg[1]/C & counter\_reg[2]/D     & 0.33       \\
        Path 12 &  0.18 &       1 &  current\_state\_reg[1]/C & counter\_reg[0]/D     & 0.32       \\
        Path 13 &  0.18 &       1 &  current\_state\_reg[1]/C & counter\_reg[1]/D     & 0.32       \\
        Path 14 &  0.19 &       0 &  z\_t\_reg[22]/C                  & z\_out\_reg[14]/D     & 0.29       \\
        Path 15 &  0.22 &       0 &  z\_t\_reg[12]/C                  & z\_out\_reg[4]/D      & 0.29       \\
        Path 16 &  0.22 &       0 &  z\_t\_reg[14]/C                  & z\_out\_reg[6]/D      & 0.30       \\
        Path 17 &  0.23 &       0 &  z\_t\_reg[15]/C                  & z\_out\_reg[7]/D      & 0.26       \\
        Path 18 &  0.23 &       0 &  z\_t\_reg[10]/C                  & z\_out\_reg[2]/D      & 0.33       \\
        Path 19 &  0.24 &       1 &  counter\_reg[0]/C                & counter\_reg[3]/D     & 0.38       \\
        Path 20 &  0.24 &       0 &  z\_t\_reg[18]/C                  & z\_out\_reg[10]/D     & 0.32       \\
        \hline
    \end{tabular}
    \caption{Table showing the main critical paths for hold-time-violation found in the implementation step.}
    \label{tab:hold_implementation}
\end{table}

\section{Utilization Report}
\begin{table}[H]
    \centering
    \small
    \captionsetup{skip=10pt} 
    \begin{tabular}{lrr}
        \hline
        Resource               & Utilization (\%) & Description \\
        \hline
        Slice LUTs             & 1.59\%           & Look-Up Tables used as logic \\
        Slice Registers        & 0.27\%           & Registers used in the design \\
        Slice                  & 1.93\%           & Total slices utilized \\
        LUT as Logic           & 1.59\%           & LUTs specifically used as logic \\
        DSPs                   & 2.50\%           & Digital Signal Processing blocks \\
        Bonded IOB             & 0.00\%           & Bonded Input/Output Blocks \\
        BUFGCTRL               & 0.00\%           & Global Clock Buffers \\
        \hline
    \end{tabular}
    \caption{Resource utilization for the CORDIC design (only non-zero values shown)}
    \label{tab:cordic_resource_utilization}
\end{table}


\section{Power Report}
\begin{figure}[H]
    \centering
    \captionsetup{skip=10pt} 
    \includegraphics[width=\textwidth]{./images/Vivado/power_report.png}
    \caption{Vivado Power Report after implementation.}
    \label{fig:power_report}
\end{figure}
The power report highlights a total on-chip power consumption of 0.095 W, with 95\% attributed to static power (0.090 W) and 5\% to dynamic power (0.05 W). The dynamic power is distributed among different components: clocks (15\%), signals (31\%), logic (33\%), DSP (21\%), with signals and logic being the highest consumers of dynamic power.

The high value of static power over dynamic power is largely to be attributed to the fact that the design is using very little of the available hardware. With reference to the Utilization Report in Chapter \ref{chap:utlization_report}, resources like Slice LUTs and Registers are used only for 1.59\% and 0.27\%, this makes the overall switching activity of the chip very small.

\section{Conclusions}
During the implementation phase, various warnings were observed. These warnings arise from the fact that the design was developed in an out-of-context mode, which is standard practice for proof-of-concept implementations.

\begin{tcolorbox}[colback=yellow!20, colframe=yellow!50!black, title=Warnings]
\begin{itemize}
    \item \textbf{DRC 23-814} Not all possible (connectivity-based) DRCs may have been run because this design is seen as Out of Context.
    \item \textbf{Route 35-197} Clock port \texttt{"clk"} does not have an associated \texttt{HD.CLK\_SRC}. Without this constraint, timing analysis may not be accurate, and upstream checks cannot be done to ensure correct clock placement.
    \item \textbf{Route 35-198} Port \texttt{"rst"} does not have an associated \texttt{HD.PARTPIN\_LOCS}, which will prevent the partial routing of the signal \texttt{"rst"}. Without this partial route, timing analysis to/from this port will not be accurate, and no routing information for this port can be exported.
    \item \textbf{Timing 38-242} The property \texttt{HD.CLK\_SRC} of clock port \texttt{"clk"} is not set. In out-of-context mode, this prevents timing estimation for clock delay/skew.
\end{itemize}
\end{tcolorbox}

In our particular study case presented in Section \ref{sec:algorithm_implementation}, these warnings are not a cause for concern. They arise because the design was intentionally tested in isolation, by just applying a single clock constraint with a 20MHz frequency. As a result, issues like incomplete connectivity checks and timing inaccuracies are reported.

Moreover, applying I/O pin constraints or testing the design with external device constraints would be meaningless in this context. As discussed in \ref{sec:algorithm_implementation}, this device is intended to be an onboard accelerator for other devices. Its primary purpose is to function as part of a larger system, and its inputs and outputs will be internally routed. Therefore, testing with external I/O constraints would not provide any meaningful insights.
\\\\
By observing Table \ref{tab:setup_implementation}  we can notice that the Worst Negative Slack (WNS) for setup timing is reported as 10.194 ns. 


\cadm{need to draw conclusions on the results obtained}
\chapter{Final considerations}
The design process adopted in this work demonstrates an efficient approach to implementing the CORDIC algorithm. The iterative methodology ensures high accuracy while maintaining low resource utilization, as confirmed by the Vivado synthesis and power analysis reports. 

It is important to note that while the implemented circuit is tailored for the constraints of the Zybo board, the modularity of the design makes it adaptable to other hardware platforms. The testing and verification phase, which included testbenches and graphical analysis, validated the functionality and robustness of the system across a wide range of input conditions.

\nocite{*}
\printbibliography

\end{document}
% -----------------------------------------------------------------
