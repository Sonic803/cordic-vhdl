\chapter{Vivado results}

\section{Critical Path}
\begin{figure}[H]
    \centering
    \includegraphics[width=\textwidth]{./images/Vivado/setup_synthesis.pdf}
    \caption{Figure showing the elaborated RTL design with the main critical paths for set-up-time-violation found in synthesis step highlighted in blue.}
    \label{fig:setup_synthesis}
\end{figure}

\begin{table}[ht]
    \centering
    \small
    \captionsetup{skip=10pt} 
    \begin{tabular}{lrrrrrrr}
        \hline
        Name &  Slack &  Levels &  Routes & From      & To                 & Total Delay    \\
        \hline
        Path 1 &   9.30 &      12 &       6 & ARG2/CLK  & x\_out\_reg[15]/D  &        10.55  \\
        Path 2 &   9.33 &      11 &       6 & ARG2/CLK  & x\_out\_reg[12]/D  &        10.52  \\
        Path 3 &   9.45 &      10 &       6 & ARG2/CLK  &  x\_out\_reg[8]/D  &        10.40  \\
        Path 4 &   9.51 &      12 &       6 & ARG2/CLK  & x\_out\_reg[14]/D  &        10.33  \\
        Path 5 &   9.56 &       9 &       6 & ARG2/CLK  &  x\_out\_reg[4]/D  &        10.28  \\
        Path 6 &   9.57 &      11 &       6 & ARG2/CLK  & x\_out\_reg[11]/D  &        10.28  \\
        Path 7 &   9.62 &      12 &       6 & ARG2/CLK  & x\_out\_reg[13]/D  &        10.23  \\
        Path 8 &   9.62 &      11 &       6 & ARG2/CLK  & x\_out\_reg[10]/D  &        10.23  \\
        Path 9 &   9.67 &       8 &       6 & ARG2/CLK  &  x\_out\_reg[0]/D  &        10.18  \\
        Path 10 &   9.69 &      10 &       6 & ARG2/CLK  &  x\_out\_reg[7]/D  &       10.16  \\
        \hline
    \end{tabular}
    \caption{Table showing data of the main critical paths found in synthesis step}
    \label{tab:setup_synthesis}
\end{table}
    

\begin{figure}[H]
    \centering
    \includegraphics[width=\textwidth]{./images/Vivado/hold_synthesis.pdf}
    \caption{Figure showing the elaborated RTL design with the main critical paths for hold-time-violation found in synthesis step highlighted in blue.}
    \label{fig:hold_synthesis}
\end{figure}  

\begin{table}[ht]
    \centering
    \small
    \captionsetup{skip=10pt} 
    \begin{tabular}{lrrrrrr}
        \hline
        Name    & Slack & Levels & Routes  & From           & To               & Total Delay \\
        \hline
        Path 11 & 0.16  & 0      & 1       & z\_t\_reg[23]/C & z\_out\_reg[15]/D & 0.29       \\
        Path 12 & 0.16  & 0      & 1       & z\_t\_reg[8]/C  & z\_out\_reg[0]/D  & 0.29       \\
        Path 13 & 0.16  & 0      & 1       & z\_t\_reg[18]/C & z\_out\_reg[10]/D & 0.29       \\
        Path 14 & 0.16  & 0      & 1       & z\_t\_reg[19]/C & z\_out\_reg[11]/D & 0.29       \\
        Path 15 & 0.16  & 0      & 1       & z\_t\_reg[20]/C & z\_out\_reg[12]/D & 0.29       \\
        Path 16 & 0.16  & 0      & 1       & z\_t\_reg[21]/C & z\_out\_reg[13]/D & 0.29       \\
        Path 17 & 0.16  & 0      & 1       & z\_t\_reg[22]/C & z\_out\_reg[14]/D & 0.29       \\
        Path 18 & 0.16  & 0      & 1       & z\_t\_reg[9]/C  & z\_out\_reg[1]/D  & 0.29       \\
        Path 19 & 0.16  & 0      & 1       & z\_t\_reg[10]/C & z\_out\_reg[2]/D  & 0.29       \\
        Path 20 & 0.16  & 0      & 1       & z\_t\_reg[11]/C & z\_out\_reg[3]/D  & 0.29       \\
        \hline
    \end{tabular}
    \caption{Table showing the characteristics regarding critical paths for hold-time-violation found in synthesis step.}
    \label{tab:hold_synthesis}
\end{table}


\begin{table}[ht]
    \centering
    \small
    \captionsetup{skip=10pt} 
    \begin{tabular}{lrrrrrr}
        \hline
        Name    & Slack & Levels & Routes & From      & To                 & Total Delay \\
        \hline
        Path 1  & 8.90  & 10     & 3      & ARG2/CLK  & x\_out\_reg[9]/D   & 10.99       \\
        Path 2  & 8.93  & 10     & 3      & ARG2/CLK  & x\_out\_reg[12]/D  & 10.96       \\
        Path 3  & 8.99  & 11     & 3      & ARG2/CLK  & x\_out\_reg[13]/D  & 10.86       \\
        Path 4  & 9.08  & 11     & 3      & ARG2/CLK  & x\_out\_reg[15]/D  & 10.82       \\
        Path 5  & 9.10  & 9      & 3      & ARG2/CLK  & x\_out\_reg[8]/D   & 10.84       \\
        Path 6  & 9.10  & 9      & 3      & ARG2/CLK  & x\_out\_reg[6]/D   & 10.84       \\
        Path 7  & 9.14  & 10     & 3      & ARG2/CLK  & x\_out\_reg[10]/D  & 10.71       \\
        Path 8  & 9.15  & 9      & 3      & ARG2/CLK  & x\_out\_reg[7]/D   & 10.74       \\
        Path 9  & 9.18  & 11     & 3      & ARG2/CLK  & x\_out\_reg[14]/D  & 10.72       \\
        Path 10 & 9.22  & 10     & 3      & ARG2/CLK  & x\_out\_reg[11]/D  & 10.63       \\
        \hline
    \end{tabular}
    \caption{Table showing the main critical paths for the setup-time-violation during the implementation step.}
    \label{tab:setup_implementation}
\end{table}


\begin{table}[ht]
    \centering
    \small
    \captionsetup{skip=10pt} 
    \begin{tabular}{lrrrrrr}
        \hline
        Name    & Slack & Levels & Routes  & From                              & To                    & Total Delay \\
        \hline
        Path 11 & 0.17  & 1      & 1       & counter\_reg[0]/C                & counter\_reg[2]/D     & 0.31       \\
        Path 12 & 0.17  & 1      & 1       & counter\_reg[0]/C                & counter\_reg[1]/D     & 0.31       \\
        Path 13 & 0.18  & 1      & 1       & counter\_reg[0]/C                & counter\_reg[3]/D     & 0.31       \\
        Path 14 & 0.22  & 0      & 1       & z\_t\_reg[19]/C                  & z\_out\_reg[11]/D     & 0.25       \\
        Path 15 & 0.22  & 0      & 1       & z\_t\_reg[23]/C                  & z\_out\_reg[15]/D     & 0.27       \\
        Path 16 & 0.23  & 0      & 1       & z\_t\_reg[8]/C                   & z\_out\_reg[0]/D      & 0.26       \\
        Path 17 & 0.25  & 0      & 1       & z\_t\_reg[12]/C                  & z\_out\_reg[4]/D      & 0.35       \\
        Path 18 & 0.25  & 0      & 1       & z\_t\_reg[22]/C                  & z\_out\_reg[14]/D     & 0.33       \\
        Path 19 & 0.26  & 1      & 1       & FSM\_reg[0]/C                    & FSM\_reg[0]/D         & 0.35 \\
        Path 20 & 0.27  & 0      & 1       & z\_t\_reg[18]/C                  & z\_out\_reg[10]/D     & 0.37       \\
        \hline
    \end{tabular}
    \caption{Table showing the main critical paths for hold-time-violation found in the implementation step.}
    \label{tab:hold_implementation}
\end{table}

\cadm{need to comment a bit these results, for example the paths for setup-time-violation correspond exactly between synthesis and implementation, whereas hold-time-violation are different, these means that place and route had a significant impact in determining these values. }

\cadm{The DRC actually gives us 6 warnings, it tells us to pipeline the multiplier block in order to enhance performances and power consumption.
But we adopted a CORDIC approach with loopbacks, that converges in 16 steps, 17 in the worst case when we have the fix step, 
inserting a register in between would cause higher delays in order to get the result. }


\section{Utilization Report}
\begin{table}[ht]
    \centering
    \small
    \captionsetup{skip=10pt} 
    \begin{tabular}{lrr}
        \hline
        Resource               & Utilization (\%) & Description \\
        \hline
        Slice LUTs             & 2.05\%           & Look-Up Tables used as logic \\
        Slice Registers        & 0.32\%           & Registers used in the design \\
        Slice                  & 2.30\%           & Total slices utilized \\
        LUT as Logic           & 2.05\%           & LUTs specifically used as logic \\
        DSPs                   & 2.50\%           & Digital Signal Processing blocks \\
        Bonded IOB             & 68.00\%          & Bonded Input/Output Blocks \\
        BUFGCTRL               & 3.13\%           & Global Clock Buffers \\
        \hline
    \end{tabular}
    \caption{Resource utilization for the CORDIC design (only non-zero values shown)}
    \label{tab:cordic_resource_utilization}
\end{table}


\section{Power Report}
\begin{figure}[H]
    \centering
    \captionsetup{skip=10pt} 
    \includegraphics[width=\textwidth]{./images/Vivado/power_report.png}
    \caption{Vivado Power Report after implementation.}
    \label{fig:power_report}
\end{figure}