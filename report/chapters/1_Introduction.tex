\chapter{Introduction}
\begin{center}
    \begin{circuitikz}
        % blocks
        \node[draw, rectangle, minimum width=6cm, minimum height=4cm, align=center] 
        (cordic) at (0,0) {CORDIC};
        
        \node[draw, rectangle, minimum width=2.5cm, minimum height=1cm, align=center]
        (lut) at (1.5,-1.2) {LUT \\ \(\arctan\)};
    
        % in
        \node[left] at ($(cordic.west) + (-1,1.5)$) {x};
        \draw[->] ($(cordic.west) + (-1,1.5)$) -- ++(1,0);
    
        \node[left] at ($(cordic.west) + (-1,0.75)$) {y};
        \draw[->] ($(cordic.west) + (-1,0.75)$) -- ++(1,0);
    
        \node[left] at ($(cordic.west) + (-1,0)$) {clk};
        \draw[->] ($(cordic.west) + (-1,0)$) -- ++(1,0);
    
        \node[left] at ($(cordic.west) + (-1,-0.75)$) {rst};
        \draw[->] ($(cordic.west) + (-1,-0.75)$) -- ++(1,0);
    
        \node[left] at ($(cordic.west) + (-1,-1.5)$) {start};
        \draw[->] ($(cordic.west) + (-1,-1.5)$) -- ++(1,0);
    
        % out
        \node[right] at ($(cordic.east) + (1,1)$) {\(\rho\)};
        \draw[->] ($(cordic.east) + (0,1)$) -- ++(1,0);
    
        \node[right] at ($(cordic.east) + (1,0)$) {\(\theta\)};
        \draw[->] ($(cordic.east) + (0,0)$) -- ++(1,0);
    
        \node[right] at ($(cordic.east) + (1,-1)$) {valid};
        \draw[->] ($(cordic.east) + (0,-1)$) -- ++(1,0);
    \end{circuitikz}
\end{center}

\vspace{10pt}
It is required to design a digital circuit for implementing the transformation from cartesian coordinates into polar coordinates using the CORDIC algorithm in Vectoring mode. It is implemented
with these recursive equations:
\begin{align*}
    x_{i+1} & = x_i - y_i \cdot d_i \cdot 2^{-i} \\
    y_{i+1} & = y_i + x_i \cdot d_i \cdot 2^{-i} \\
    z_{i+1} & = z_i - d_i \cdot \arctan(2^{-i})
\end{align*}
where \(d_i = -1\) if \(y_i > 0\), \(+1\) otherwise. After \(n\) iterations, the equations converge to:
\[
    x_n = A_n \cdot \sqrt{x_0^2 + y_0^2}
\]
\[
    y_n = 0
\]
\[
    z_n = z_0 + \arctan\left(\frac{y_0}{x_0}\right)
\]
\[
    A_n = \prod_{i=0}^n \sqrt{1 + 2^{-2i}}
\]
\section{Circuit Applications}
The CORDIC (COordinate Rotation DIgital Computer) algorithm is an iterative method for performing vector rotations and 
solving mathematical functions such as trigonometric, hyperbolic, exponential, logarithmic, and square root operations. 
It was introduced by Jack E. Volder in 1959 to simplify the computation of these functions in hardware with limited resources.
\\\\
CORDIC is widely used because it eliminates the need for multiplication and division, relying instead on shift and addition operations. 
This makes it highly efficient for hardware implementations, particularly in devices with limited computational power, 
such as embedded systems, calculators, and digital signal processors (DSPs).
\\\\
For example the Intel 8087, a floating-point coprocessor introduced in the early 1980s, utilized the CORDIC algorithm to perform 
efficient trigonometric and hyperbolic computations, such as sine, cosine, and arctangent, without relying on hardware multipliers.
Similarly, during the Apollo Lunar Module missions, a precursor concept to CORDIC was employed in the Apollo Guidance Computer (AGC) 
to perform real-time navigation calculations. The AGC used iterative methods to determine angles and distances for lunar landings, 
efficiently converting Cartesian spacecraft coordinates to polar forms to ensure precise trajectory adjustments during descent.

The CORDIC algorithm can operate in two different modes: \textbf{rotation mode} and \textbf{vectoring mode}
\begin{itemize}
    \item \textbf{Rotation mode:} rotates a vector by a specified angle, used for calculating trigonometric functions or vector transformations
    \item \textbf{Vectoring mode:} the input vector will be rotated towards the x-axis until the y-component is reduced to zero. After a fixed number of iterations, the modulus of the vector can be retrieved on the x-axis, while its phase corresponds to the angle of rotation.
\end{itemize}
In this context, this project will focus on developing the vectoring mode of the CORDIC algorithm to convert Cartesian coordinates into polar form.